\documentclass{article}
\usepackage{graphicx} % Required for inserting images

\title{Exploring and optimising pandemic response policies with a stylised agent-based model}
\author{Jeonghwa Kang, Juste Raimbault}
\date{March 2023}

\begin{document}

\maketitle


% abstract
% In recent years, the world has been affected by the massive spread of COVID-19 without knowing when it will end. By the time COVID-19 broke out, most countries did not have proper preparedness when the number of infected cases had started rising. This research combined an agent-based model with a compartmental model known as the SIRV to study the spread and control of epidemics. The model is used to study the effectiveness of different vaccination rates, infectious periods, immunity periods and social distancing levels in controlling epidemics during the first year of the outbreak. This research's primary analysis and learning methods include the NSGAII for parameter optimisation, Saltelli global sensitivity analysis for impact assessment of the input parameters, linear regression, and a random forest for forecasting the parameter outcomes. Also, the random sampling methods used in the analysis involve the LHS and Direct Sampling. As a result, all the model's target input parameters significantly impacted the overall infected cases. Among the input parameters, the infectious period had the most impact on the spread of infectious diseases. A longer infectious period greatly stimulated the spread of infectious diseases. In contrast, the rest of the input parameters, including the vaccination rate, immunity period and social distancing level, indicated that the higher the values, the better the suppression of epidemics.



\section{Introduction}


The impact of one of the fatal epidemics in history, the Black Death, has led to the death of almost one-third of the population of Europe (Editors, 2010), and the most recent epidemic in the world today, COVID-19, is currently costing millions of valuable human lives. Before the introduction of mathematical and computational modelling, humans had little knowledge of effectively controlling and minimising the spread of infectious diseases. Monitoring and analysing the impacts of preventive measures on the spread of epidemics had to be utterly dependent on assessing the aftermath, which is the damage done to a population. Due to this, developing and applying the countermeasures often took a tremendous amount of time. However, recent advances in mathematical and computational modelling began to play an essential role in forecasting epidemics by hypothetically simulating agents’ interactions to understand complex spatial behaviours. Forecasting disease transmission dynamics provides ways for humans to develop and apply appropriate preventive measures and ensure the adequate use and distribution of limited material and human resources. Furthermore, the importance of forecasting the spread of diseases in recent years has become even more crucial as the world is rapidly globalising and becoming closer with ever-advancing technologies that make it easier for humans to travel faster and further around the globe.






\end{document}
