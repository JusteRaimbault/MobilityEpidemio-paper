\documentclass[smallextended]{svjour3}       % onecolumn (second format)
%\documentclass[twocolumn]{svjour3}          % twocolumn

\smartqed  % flush right qed marks, e.g. at end of proof

\usepackage{graphicx}

\begin{document}

\title{Exploring and optimising pandemic response policies with a stylised agent-based model}
%\subtitle{Do you have a subtitle?\\ If so, write it here}

%\titlerunning{Short form of title}        % if too long for running head

\author{Jeonghwa Kang          \and
        Juste Raimbault
}


\institute{J. Kang \at
              CASA, UCL \\
           \and
           J. Raimbault \at
              LASTIG, IGN-ENSG\\
              \email{juste.raimbault@ign.fr}
}

%\date{Received: date / Accepted: date}
% The correct dates will be entered by the editor


\maketitle

\begin{abstract}
% In recent years, the world has been affected by the massive spread of COVID-19 without knowing when it will end. By the time COVID-19 broke out, most countries did not have proper preparedness when the number of infected cases had started rising. This research combined an agent-based model with a compartmental model known as the SIRV to study the spread and control of epidemics. The model is used to study the effectiveness of different vaccination rates, infectious periods, immunity periods and social distancing levels in controlling epidemics during the first year of the outbreak. This research's primary analysis and learning methods include the NSGAII for parameter optimisation, Saltelli global sensitivity analysis for impact assessment of the input parameters, linear regression, and a random forest for forecasting the parameter outcomes. Also, the random sampling methods used in the analysis involve the LHS and Direct Sampling. As a result, all the model's target input parameters significantly impacted the overall infected cases. Among the input parameters, the infectious period had the most impact on the spread of infectious diseases. A longer infectious period greatly stimulated the spread of infectious diseases. In contrast, the rest of the input parameters, including the vaccination rate, immunity period and social distancing level, indicated that the higher the values, the better the suppression of epidemics.
\keywords{First keyword \and Second keyword \and More}
\end{abstract}







\section{Introduction}

The impact of one of the fatal epidemics in history, the Black Death, has led to the death of almost one-third of the population of Europe, and the most recent epidemic in the world today, COVID-19, is currently costing millions of valuable human lives. Before the introduction of mathematical and computational modelling, humans had little knowledge of effectively controlling and minimising the spread of infectious diseases. Furthermore, the importance of forecasting the spread of diseases in recent years has become even more crucial as the world is rapidly globalising and becoming closer with ever-advancing technologies that make it easier for humans to travel faster and further around the globe \cite{saker2004globalization}.

The first modelling approach to disease spreading was in the 17th century as John Graunt  conducted an empirical study on infections affecting individuals in various regions in Britain \cite{morabia2013epidemiology}.
%Daniel Bernoullie, in the 18th century, developed a more sophisticated model, which is more of a data-driven and equation-based approach to studying the outbreak of the smallpox epidemic that happened in Europe (Dietz and Heesterbeek, 2002). In the 19th century, various mathematical equational models were further developed for modelling the complex spread of infectious diseases. The well-known equational models include the Ordinary Differential Equations (ODEs), Partial Differential Equations (PDEs) and Differential Equations. In recent years, Agent-Based Modeling (ABMs) has combined traditional equational models to study complex spatial dynamics of epidemics. After several devastating epidemic outbreaks, particularly the SARS, MERS, and COVID-19, in the recent 21st century, many models have played a vital role in developing essential government policies to counter the epidemic outbreaks from their emergence in a particular country to the current global pandemic. For example, the early spread of the COVID-19 pandemic has led to travel restrictions and closures of the countries’ borders. Monitoring the impact of various preventive measures, such as social distancing or movement policies, has been crucial in evaluating these decisions. However, strict long-term restriction policies had other devastating impacts on local and global economies. Economic evaluations and government policies have combined both economic and epidemic models to assess the economic consequences of COVID-19, informing policy calls for easing restrictions on a situational basis. At the same time, the models for social contact and mobility, have provided practical ways to evaluate appropriate pathways to reduce regulations for mobility and distancing measures safely. Finally, the advancement in agent-based modellings has provided flexibility in evaluating the dynamics of epidemics under various complex scenarios. These include examples of how governments can achieve mass immunisation or suppression of rapidly growing epidemics by introducing different vaccination strategies. Today, the methods of infectious morbidity forecasting were able to advance quickly due to the recent deployment of information supervision systems and a vast volume of statistics available for analysis. Out of all, human interaction is the most crucial factor which plays a significant role in transmitting the virus from one to the other. Interestingly, its dynamic can be modelled via a combination of ABM and compartmental model representing Spatio-temporal features in different stages of the agents during the infection process, e.g., an ABM composed of the Susceptible-Infected-Recovered (SIR) or the Susceptible-Exposed-Infected-Recovered models (SEIR) compartmental model, to name a few.

% research question
%The main objective in keeping humans safe against fatal disease outbreaks is to control the spread of infectious diseases successfully. The first step to successfully controlling the spread of diseases would be to identify which factors are likely to impact the dynamics of the epidemics. For instance, we might be interested in the effects of social distancing and wearing masks since they are well-known preventive measures that effectively mitigate the virus’s spread (Kwon et al., 2021). Furthermore, we might also be interested in the effects of vaccination since the high rate of immunisation is known to successfully control the spread of diseases and even achieve herd immunity by mass immunisation (Bicher et al., 2022). This research is generic and stylizedfor studying the spread and control of contagious respiratory illnesses such as COVID-19 and Influenza. Because the model is rather generic, it is flexible in simulating various scenarios that consider different types of infectious diseases with particular social and geographical contexts.

% This leaves us with several important research questions to answer.
%What factors affect the spread of infectious diseases, and how can we estimate the impacts of those factors?
%To what extent do different preventive measures protect humans against the spread of infectious diseases?
%To what extent do different longevity of immunity and infectious periods of the human body affect the spread of infectious diseases?
%To what extent do the optimal values of the chosen factors control the spread of infectious diseases during the first year of the outbreak?

%Questions like these are worth considering as they often come into play when recognising the importance of modelling epidemics. Guided by these research questions, there are several objectives which we list according to the analytical steps in this research:
%1. Implement an ABM that adapts the compartmental model’s dynamics to monitor and track the spread of infectious diseases in a hypothetical scenario.
%2. By using the LHS for generating random values of the model’s input parameters, explore the output space of the model by analysing the statistical distributions.
%3. Using the Grid Sampling to evaluate every possible combination of the model’s input parameter values, identify which combination results in the highest and the lowest proportional numbers of sick agents.
%4. Using the Saltelli sensitivity analysis, calculate sensitivity indices for each model input parameter.
%5. By using NSGAII multi-criteria calibration method, optimise the model input parameter values relative to the proportional number of sick agents.

%1.3 Project Scope
%This work aims to analyse and quantify the impacts of the chosen input parameters of the model on the spread of infectious diseases. The simulation projects a real-life situation where agents commute from home to their designated workplaces via different modes of transportation on a fixed daily schedule. At the beginning of the simulation, the entire population is set to susceptible. Then, an infected agent is introduced daily for the first 30 days because if none of the infected agents gets introduced into the environment, the infection process does not happen. The model’s output parameters include the number of agents associated within each state of the compartmental model upon which the ABM is based. Since the ABM is based on the SIRV compartmental model, proportional numbers of susceptible, infected, recovered, and vaccinated agents have been collected accordingly. Several analysis methods used in this research include the NGSAII for parameter optimisation (Pareto Front) and the Saltelli global sensitivity analysis for calculating sensitivity indices for the model’s input parameters. The LHS and Direct Sampling methods based on the uniform distribution are used for randomised sample selection.
%1.4 Ethical Consideration
%This work does not handle any real-life data available to the public. There will be no data concerning a real person's identity, nor does the 2D environment depict confidential areas. All the data and predictions are obtained after the completion of the simulation. Since data and predictions generated from this work can be used for the public good to design choices that may facilitate more effective control of epidemics, the data analysis process must be as transparent as possible without any bias.


% Understanding the Importance of Epidemic Forecasting
%Accurate analysis of disease transmission is crucial in preserving the global economy and health since disease outbreaks play an essential role in worldwide morbidity and mortality. Epidemic forecasting is vital in allowing humans to prepare for an appropriate countermeasure against disease outbreaks. However, forecasting an epidemic is not easy as it seems. There is extensive ongoing global economic damage from the spread of infectious diseases every year, primarily due to the unpredictable nature of the epidemics (Boucekkine et al., 2021). The worst part of a disease outbreak is that it costs millions of valuable human lives every year. Furthermore, advancements in multiple fields, including transportation, communication, and trading, make the world increasingly interconnected, facilitating the pandemic process (Bickley et al., 2021). Due to the recent deployment of information supervision systems and the significant amounts of statistics available for analysis, the ABM is considered the practical approach to forecasting various outcomes via hypothetical simulation. Plus, the ABM efficiently overcomes the traditional analytical and deterministic obstacle models. Once an accurate model is built, it would help understand the complex spatial dynamics of epidemics and possibly predict the outcome using a supervised machine learning algorithm such as a random forest. Furthermore, this would greatly support the work of epidemiologists in many terms. For example, analysing the impacts of epidemiological factors on disease transmission dynamics via ABM can assist epidemiologists in developing effective countermeasures against possible disease outbreaks in the future (Miksch et al., 2019).

%Dynamics of the ABM and Compartmental Model
%Chumachenko et al. (2018) define Agent-based modelling as ‘a relatively new direction in simulation, which is used to study the decentralised systems, the dynamics of functioning of which is determined not by global rules and laws.’ As the statement implies, the main advantage ABMs offer is an understanding of the unbiased dynamics of functioning. Its’ main aim is to understand these global rules based on the assumptions and interactions of the individuals in the system.
%The SIR model, which Kermack and McKendrick initially introduced in 1920, is often applied to the mathematical modelling of infectious diseases. This model for influenza describes the process of disease transmission in various states. For example, (S) stands for susceptible, (I) for infected, and (R) for recovered states of the agents in the model. Chowell, Miller and Viboud (2008) use the SEIRD compartmental model to study the spread of influenza in three countries: the U.S., France, and Australia. The SEIRD model is an advanced version of the traditional SIR model, consisting of individual agents exposed to the diseases but not exhibiting symptoms yet. Also, this model considers the death of agents in the population. To better understand the model's dynamics, the SEIRD model is demonstrated below,
%[ eqs seird]
%For example,,, and indicate the numbers of exposed, immune, and dead individual agents respectively. The shows the transmission rate, and �� is the total population. denotes the rate at which respective agents go from exposure to infection. Lastly, �� and �� indicate the recovery and mortality rates, respectively. Using this SEIRD compartmental model, Chowell, Miller and Viboud (2008) study the effectiveness of vaccination in spreading and controlling influenza. They concluded that a higher vaccination rate of susceptible agents is necessary to prevent and effectively reduce the spread of flu. Furthermore, they found that the re-vaccination process would further enable better control of the diseases. Another approach to modelling the spread of infectious diseases comes from a discrete model (Difference Equation). This model determines the variable quantities at distinct time intervals. Hence, this approach is well suited to representing the time series data. For example, the SIR compartmental model in the discrete form is shown below,
%[]
%In the equation, ���� , ���� , and ���� indicate the number of susceptible, infected, and recovered individual agents at time �� respectively. �� indicates the number of contacts and �� represents the birth and death rates. Lastly, the �� refers to the recovery rate. Ramani et al. (2004) used discrete-time models to study the spread of epidemics over time under two different scenarios. The first scenario assumes no changes in the number of total populations with no occurrence of death. The second scenario assumes that the total number of population changes either by permanent recovery or death of agents. For the first scenario, the number of infectiousdiseases reached a fixed point without disappearing. However, according to the second scenario, where individual agents are allowed to leave the population by means of death or permanent immunity, the epidemic often died out completely in the simulation.

%The models mentioned above are all examples of traditional models. Therefore, they often pose limitations compared to the most recent Agent-Based Model. The main limitation is that traditional models fail to describe agent behaviour variations accurately. To overcome the limitation, Chumachenko et al. (2018) proceeded to advance the model by combining the concept of the SIR compartmental model with an ABM to study the spatial aspect matter in propagating epidemics. This type of modelling is a relatively new study area compared to traditional modellings. The ABM’s environment comprises grid cells or patches where individual agents reside. These patches are where the agents contact each other and interact with the environment. The models’ patches and agents may have specific features. For instance, patches may have attributes that can affect agents’ movement or interactions. We can also define characteristics such as location, age, and many other factors believed necessary in the model for each agent. Once the agents are defined with their own set of characteristics, they proceed through a set of rules which enable them to interact with other groups of agents within the simulation. The agents’ range of interactions may differ depending on the model types. For instance, in predator-prey models, one kind of agent may feed off another, while in competition models, agents with entirely different properties may compete for survival. In epidemic models, infected agents may infect other susceptible agents through interaction. Furthermore, adapting the compartmental model into the ABM advances the epidemic model by adding multiple states of agents, generating much more detailed simulation results.

% The Interactions and Infections
%The characteristics of agents are important in studying and forecasting the spread of diseases as they determine how contacts and infections between the agents occur in a simulation. Chumachenko et al. (2018) define two agents: the ‘age-human’ and ‘location-human’. Firstly, the age of an agent determines the probability of contact between agents. Changes in the value of the contact rate would also affect the infection rate. There are five age groups: children, teens, youth, adults, and elderly. Among the group, both children and elderly groups are deemed to be in contact with fewer people than younger groups. Also, the communication between the agents is made if they are located on the same patch only. This is mainly to reduce the computational complexity of the model by controlling the agents’ interactions per patch instead of considering multiple neighbouring patches altogether. Reducing the computational cost for simulation is a crucial factor, especially for the ABMs, since the cost may increase when there are many combinations of the variables to consider per execution. Because the research depends on developing the ABM on Netlogo, minimising the computational complexity is crucial for the system’s smooth operation since the Netlogo modelling environment is mainly for educational purposes rather than advanced research.

%Minoza, Bongolan and Rayo (2021) introduce several essential variables that describe the agents’ characteristics, including age, status, wearing face masks, physical distancing, lockdown and immunity. Depending on the age of an agent, the probability of contact differs among the agents. The variable that checks whether a susceptible agent is wearing a mask during the interaction with an infected agent may well reflect a real-life situation. Hence, this can be a good consideration factor for calculating an accurate infection rate that can better shape the disease transmission dynamics. The immunity variable here indicates if an agent is permanently immune or not. The limitation with the SIR or SEIR model is that once an agent fully recovers from infection, the agent is considered permanently immune throughout the simulation. However, this does not correctly depict a real-life situation as immunity against most infectious
diseases is impermanent. Therefore, the immunity period of an agent should be re-constructed
in a way that is not permanent; hence, if a specific time passes, the agent becomes susceptible
and vulnerable to infection again. To go a step further, a variable that defines an agent’s
medical conditions can play an essential role in determining a reliable infection rate as it is
known that a person who is suffering from cardiovascular diseases is at a high risk of getting
infected, mainly due to a defeated immune system (Fekadu et al., 2021). The properties of
agents that can impact the infection rate should carefully be considered based on the interest
of the research topic, as they play an essential role in shaping the overall outcome of a
simulation.

% Vaccination and Immunity
Kaszowska-Mojsa, Włodarczyk and Szymańska (2022) carried out an analysis of the impact of
the COVID-19 outbreaks on labour productivity by comparing the results obtained from three
different scenarios using the dynamic stochastic general equilibrium model with an agent-based
epidemic component. The paper indicates the effectiveness of vaccines in association with
various preventive measures such as lockdown, quarantine, and social distancing against the
spread of the virus in a hypothetical simulation. Furthermore, the scenarios in this paper
introduce different levels of restrictions to control the spread of COVID-19. For instance,
scenarios with varying intensity levels of lockdown, quarantine, and social distancing with and
without vaccination in a simulation were tested, and the results were compared. The conclusive
results indicate that the vaccination process is crucial in controlling epidemics’ spread. It is
concluded by the findings that the effectiveness of the lockdown, quarantine, and social
distancing alone could not achieve herd immunity even though such procedures may
temporarily slow down the general spread of diseases. For a community to achieve herd
immunity, it has been found that approximately 90\% of the population must be vaccinated or naturally immunised from recovery. This process can be fastened if the effectiveness of
vaccines and the probability of agents receiving the vaccines increase. Physical well-being is
also a crucial factor that ensures society’s safety from harmful diseases, but the importance of
healthy economic growth cannot be neglected either. Therefore, policymaking during a
pandemic would be challenging in a situation where human lives and the economy are at stake.
As a result, the restrictions such as lockdown and quarantine will eventually get lifted, and the
preventive measures will be eased once humans know how the spread of diseases can be
controlled and prevented. In conclusion, the right approach would be to keep focusing on
evaluating the impacts of various preventive measures against such diseases and finding the
optimal solution sets that can effectively control disease spread.

% Assessing the Impacts of Model’s Input Parameters
% Baquela and Olivera (2022) utilises one of the multi-objective evolutionary algorithms (MOEA) for optimising the distribution of limited COVID-19 vaccines across the provinces in Argentina. The distribution of vaccines is prioritised among different groups of the population. The group
with the highest priority comprises the older people and medical workers, who comprise about
one-third of Argentina’s total population. Then, the overall distribution of the vaccines across
geographical groups is indicated by the average Gini coefficient. A value close to zero Gini
coefficient indicates perfect equality. In this case, having a zero Gini value would mean all the
vaccines are equally distributed throughout the provinces in Argentina. A value close to one
would mean the distribution of vaccines across the provinces in Argentina is entirely biased.
Baquela and Olivera (2022) denote that the usual way to group the population is based on the
severity of infected cases, the standard criterion most governments depend on to define
priorities for vaccination during the COVID-19 pandemic. However, this is not always the case
for countries like Argentina as there are multiple issues to consider, such as logistic issues to
distributing vaccines and legal and ethical issues. Due to these, one cannot simply supply a high
number of vaccines to some provinces while keeping other regions without vaccines.
Furthermore, vaccines are expensive, and certain countries cannot bear the costs of these
vaccines to cover their entire population. This is known to further damage the economy since
the ways to control the spread of infectious diseases in countries without vaccines are heavily dependent on isolation policies (Hafner et al., 2020). This is where multi-objective evolutionary
algorithms can help solve problems that pose many conflicting objective functions. NSGAII
applies Pareto dominance for the fitness calculation, building fronts of solutions. The Pareto
Front refers to a set of optimal solutions which are non-dominant to each other but are
superior to the rest of the solutions in the search space (Yusoff, Ngadiman and Zain, 2011).
However, the Pareto front does not identify how much variations in the model’s input
parameter affect the variations in the model output parameters. One way to tackle this
problem is by conducting sensitivity analysis which acts as an in-depth study of all the model
variables. Conducting sensitivity analysis is useful, particularly for decision-makers. The results
obtained by such analysis are far more reliable in the sense that it quantifies the impacts of the
model’s input parameters by calculating the changes in the total variance of the model’s output
parameters, thereby identifying where they can make improvements to obtain better results in
the future (Saltelli, 2008).

Wu et al. (2013) developed a model of infectious diseases based on a compartmental model
which consist of six compartments: susceptible, exposed, infected, recovered, dead and
vaccinated. The model has been parameterised to model a disease outbreak in a large
metropolitan area. These input parameters include age, contact rate, quarantine effectiveness
rate, disease infectivity rate, vaccine effectiveness rate, and fatality rate. In order to quantify
the impacts of each input parameters, Saltelli’s global sensitivity analysis is applied to the
infection model. This type of analysis is used to determine how much of the variability in model
output is dependent on each of the input parameters, either upon a single parameter or
interaction between different parameters (Zhang et al., 2015). For instance, if one of the input
parameters has a small value of the sensitivity index, it indicates that the variation of this
parameter is most likely to result in a slight variation in the output parameters. On the other
hand, if the sensitivity index is high, a change in the value of the input parameter will likely lead
to a dramatic change in the output parameter values. The sensitivity analysis can, therefore,
guide experimental focus so that researchers can take special care in obtaining more precise
and reliable measures of parameters by referring to these sensitivity indices.


As discussed above, it has proved to be quite effective in modelling the spread of infectious
diseases in various compartmental models such as the SIR and the SIRV. These models, in
general, allow for a detailed evaluation of various preventive measures such as vaccination
rates, quarantine, social distancing or lockdowns as one of the model's input parameters on the
spread of epidemics. Various compartments precisely describe each state of an agent in the
infection process and provide solutions to control the spread of epidemics effectively.
Furthermore, the advantage of using these models is that they provide a better representation
of the data because data gets collected in discrete time intervals. On the other hand, they pose
limitations too. One of the limitations is that they are not suitable for collecting Spatio-
temporal data. Hence, these models ignore the spatial aspects of the infection dynamics. The
spatial data is as significant as the temporal data in the field of epidemiology because the
infection dynamics can vary greatly depending on different spatial structures of the
environment. As a solution, the ABM can combine one of the compartmental models
mentioned earlier to overcome these limitations. Since the effectiveness of vaccination is one
of the main interests of the research, instead of adopting the traditional SIR model, we can base
the SIRV model on our ABM to consider the vaccination state of the agents in the simulation.
This way, the model collects not just the temporal data but also the spatial data as an outcome
of the simulation.
Various methods have been mentioned earlier regarding the analysis tools, including the
NSGAII for parameter optimisation and Saltelli for global sensitivity analysis, to name a few.
One thing to note is that this research should not stick to just one analysis method to assess the
overall impacts of the model's input parameters. Instead, we combine the analysis methods.
The NSGAII to obtain the optimal set of the input parameter values that best control the spread
of epidemics and Saltelli global sensitivity analysis to find out which model input parameter
significantly affects the overall output parameters. Also, training a random forest learning
algorithm is an excellent way to forecast the outcome of our ABM simulation and assess how
well the model can predict the spread of infectious diseases. Through the random forest model,
we can also calculate the importance of each input parameter and compare the result with the result obtained by Saltelli sensitivity analysis. In order to visualise the statistical distribution of
the model outputs, random sampling methods such as the LHS and direct sampling, which
select samples based on uniform distribution, can be adapted in the analysis process. Lastly,
linear regression is also proven to be an effective tool for predicting the dependent variable,
which is the proportional number of sick agents in our case. Coefficient values of the input
parameters can be calculated using this tool and estimate the impacts of each of these
parameters on the dependent variable.


\section{Model description}
The main interest of this research is investigating the effectiveness of various pre-defined model
input parameters on spreading and controlling infectious diseases that result while the agents
are commuting from home to the workplace. This model assumes that most human interactions
occur while people travel and work. Hence, the scenario is primarily about people commuting
from home to their designated workplaces via different modes of transportation, closely
depicting ordinary people's daily routine.
Using NetLogo, the model is tested on a 2D environment where agents representing people are
distributed homogeneously throughout. The simulation of this model considers several essential
assumptions. The entire duration of the simulation is set to 365 days. Each day is equivalent to
24 ticks. Hence, each tick is equivalent to an hour. A single day comprises three pre-defined
phases: the ‘Home-hour’, ‘Commute-hour’ and ‘Work-hour’ phases. Each phase is indicated by
different colours of the patches in the simulation; orange indicates the ‘Home-hour’ phase, black
for the ‘Commute-hour’ phase and cyan for the ‘Work-hour’ phase.
During these phases, the movement of agents is completely randomised. No learning behaviour
alternates the agents’ movement in the simulation since it would be difficult for an individual to
be fully aware of another agent’s infection status. During the interaction, agents fall into one of
the four states defined by the SIRV compartmental model, which this research’s ABM is based
upon. The SIRV model is an extended SIR model that accounts for the vaccination of the
susceptible population (Poonia et al., 2022).
Agents can interact with each other based on the state in which they fall. However, the
interaction and infection processes are set to occur between the agents within the same patch
only. The infection does not happen during the 'Home-hour' phase as this model does not
consider social activities or household infections. During the 'Commute-hour' phase, an infected
agent can only infect the susceptible agents that use the same mode of transportation as itself.
It would be logically sensible that an infected agent commuting via train can only interact with and infect other susceptible agents commuting via the same mode of transportation. This same
logic applies during the 'Work-hour' phase, where an infected agent can only interact with and
infect other susceptible agents working in the same company. The following table lists all the
modes of transportation and the names of companies defined in our model.
[Table 1]
The actual company names are used here to assign each agent’s company variable. There are
total of fifteen companies which are randomly categorised into small, mid, and big-sized
companies. The central assumption is that the bigger the company size, the more the human
interactions involved. Hence, the agents working in a big-sized companies are more likely to get
infected than the agents working in mid and small-sized companies.
At the beginning of the simulation, 1500 susceptible agents are randomly distributed in the
simulation environment. The variable values for the agents, such as the transportation and
company, are randomly assigned during the setup stage. An infected agent is then constantly
introduced at a random location for the simulation's first month (30 days). Every day, a random
percentage value from zero to one percent of the total population gets replaced. This is based on
the assumption that the population is dynamic. A certain number of people leave the area, and
others come to stay in the area. Hence, the total population is not fixed to the initial population
throughout the simulation.


\subsection{Model parametrisation}

Infection rates are calculated based on the values of the global, patch and turtle variables. The
variables that define contact rates for each mode of transportation and company are pre-defined
with specific values, which are shown below.
%[Global variables]
%Contact rates by different modes of transportation:
%Variable NameContact Rate
%contact_rate_bus40 %
%contact_rate_tube35 %
%contact_rate_train30 %
%contact_rate_walk10 %
%contact_rate_car0%
%T ABLE 2. CONTACT RATES FOR EACH OF THE TRANSPORTATION M ODES
%Contact rates by different sizes of companies:
%Variable Name
%Big-sized Company
%Mid-sized Company
%Small-sized Company
%Contact Rate
%40 %
%35 %
%30 %
%T ABLE 3. CONTACT RATES FOR EACH OF THE C OMPANY SIZES


\bibliographystyle{spphys}
\bibliography{biblio}

\end{document}
